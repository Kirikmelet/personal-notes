% Options for packages loaded elsewhere
\PassOptionsToPackage{unicode}{hyperref}
\PassOptionsToPackage{hyphens}{url}
%
\documentclass[
]{article}
\usepackage{amsmath,amssymb}
\usepackage{iftex}
\ifPDFTeX
  \usepackage[T1]{fontenc}
  \usepackage[utf8]{inputenc}
  \usepackage{textcomp} % provide euro and other symbols
\else % if luatex or xetex
  \usepackage{unicode-math} % this also loads fontspec
  \defaultfontfeatures{Scale=MatchLowercase}
  \defaultfontfeatures[\rmfamily]{Ligatures=TeX,Scale=1}
\fi
\usepackage{lmodern}
\ifPDFTeX\else
  % xetex/luatex font selection
\fi
% Use upquote if available, for straight quotes in verbatim environments
\IfFileExists{upquote.sty}{\usepackage{upquote}}{}
\IfFileExists{microtype.sty}{% use microtype if available
  \usepackage[]{microtype}
  \UseMicrotypeSet[protrusion]{basicmath} % disable protrusion for tt fonts
}{}
\makeatletter
\@ifundefined{KOMAClassName}{% if non-KOMA class
  \IfFileExists{parskip.sty}{%
    \usepackage{parskip}
  }{% else
    \setlength{\parindent}{0pt}
    \setlength{\parskip}{6pt plus 2pt minus 1pt}}
}{% if KOMA class
  \KOMAoptions{parskip=half}}
\makeatother
\usepackage{xcolor}
\setlength{\emergencystretch}{3em} % prevent overfull lines
\providecommand{\tightlist}{%
  \setlength{\itemsep}{0pt}\setlength{\parskip}{0pt}}
\setcounter{secnumdepth}{-\maxdimen} % remove section numbering
\ifLuaTeX
\usepackage[bidi=basic]{babel}
\else
\usepackage[bidi=default]{babel}
\fi
\babelprovide[main,import]{japanese}
% get rid of language-specific shorthands (see #6817):
\let\LanguageShortHands\languageshorthands
\def\languageshorthands#1{}
\ifLuaTeX
  \usepackage{selnolig}  % disable illegal ligatures
\fi
\IfFileExists{bookmark.sty}{\usepackage{bookmark}}{\usepackage{hyperref}}
\IfFileExists{xurl.sty}{\usepackage{xurl}}{} % add URL line breaks if available
\urlstyle{same}
\hypersetup{
  pdftitle={多項式のグラフ},
  pdflang={ja},
  hidelinks,
  pdfcreator={LaTeX via pandoc}}

\title{多項式のグラフ}
\author{}
\date{}

\begin{document}
\maketitle

\hypertarget{ux591aux9805ux5f0fux306eux30b0ux30e9ux30d5}{%
\section{多項式のグラフ}\label{ux591aux9805ux5f0fux306eux30b0ux30e9ux30d5}}

英語: Polynomial Graphs

\begin{quote}
\textbf{注意!} これらが分かるを必要がある!

\begin{itemize}
\tightlist
\item
  多項式の終了動作
\item
  多項式のゼロ
\end{itemize}
\end{quote}

\hypertarget{ux591aux9805ux5f0fux95a2ux6570ux3092ux5206ux6790ux3059ux308b}{%
\subsection{多項式関数を分析する}\label{ux591aux9805ux5f0fux95a2ux6570ux3092ux5206ux6790ux3059ux308b}}

この例えばに、我らはこの関数を使う。 {\[f(x) = (3x - 2)(x + 2)^{2}\]}

\hypertarget{yux5207ux7247ux3092ux898bux3064ux3051ux308b}{%
\subsubsection{\texorpdfstring{{\(y\)}切片を見つける}{y切片を見つける}}\label{yux5207ux7247ux3092ux898bux3064ux3051ux308b}}

{\(f\)}グラフの{\(y\)}切片を見つけるの為に我らは{\(f(0)\)}を見つける。

\[f(0) = (3(0) - 2)(0 + 2)^{2}\]

\[f(0) = ( - 2)(4)\]

\[f(0) = - 8\]

{\(y = f(x)\)}のグラフの{\(y\)}切片は{\((0, - 8)\)}

\hypertarget{xux5207ux7247ux3092ux898bux3064ux3051ux308b}{%
\subsubsection{\texorpdfstring{{\(x\)}切片を見つける}{x切片を見つける}}\label{xux5207ux7247ux3092ux898bux3064ux3051ux308b}}

{\(x\)}切片を見つけるの為に、我らこの公式を解く、{\(f(x) = 0\)}

\[0 = (3x - 2)(x + 2)^{2}\]

今は我らが零積性質を使う。

\hypertarget{ux3064}{%
\paragraph{1つ}\label{ux3064}}

{\[3x - 2 = 0\]}{\[x = \frac{2}{3}\]}

\hypertarget{ux3064-1}{%
\paragraph{2つ}\label{ux3064-1}}

{\[x + 2 = 0\]}{\[x = - 2\]}{\(y = f(x)\)}のグラフの{\(x\)}軸は{\((\frac{2}{3},0)\)}と{\(( - 2,0)\)}
我らの働くを見つけった、その{\(\frac{2}{3}\)}は1つの重根もう{\(- 2\)}は2つの重根です。

又、グラフが{\(x\)}軸{\((\frac{2}{3},0)\)}と{\(( - 2,0)\)}であるを横切る。

\hypertarget{ux7d42ux4e86ux52d5ux4f5cux3092ux898bux3064ux3051ux308b}{%
\subsubsection{終了動作を見つける}\label{ux7d42ux4e86ux52d5ux4f5cux3092ux898bux3064ux3051ux308b}}

終了動作を見つけるの為に、我らは関数の標準形であるのとき、主な項を解く。

\end{document}
