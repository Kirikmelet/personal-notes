% Options for packages loaded elsewhere
\PassOptionsToPackage{unicode}{hyperref}
\PassOptionsToPackage{hyphens}{url}
%
\documentclass[
]{article}
\usepackage{amsmath,amssymb}
\usepackage{iftex}
\ifPDFTeX
  \usepackage[T1]{fontenc}
  \usepackage[utf8]{inputenc}
  \usepackage{textcomp} % provide euro and other symbols
\else % if luatex or xetex
  \usepackage{unicode-math} % this also loads fontspec
  \defaultfontfeatures{Scale=MatchLowercase}
  \defaultfontfeatures[\rmfamily]{Ligatures=TeX,Scale=1}
\fi
\usepackage{lmodern}
\ifPDFTeX\else
  % xetex/luatex font selection
\fi
% Use upquote if available, for straight quotes in verbatim environments
\IfFileExists{upquote.sty}{\usepackage{upquote}}{}
\IfFileExists{microtype.sty}{% use microtype if available
  \usepackage[]{microtype}
  \UseMicrotypeSet[protrusion]{basicmath} % disable protrusion for tt fonts
}{}
\makeatletter
\@ifundefined{KOMAClassName}{% if non-KOMA class
  \IfFileExists{parskip.sty}{%
    \usepackage{parskip}
  }{% else
    \setlength{\parindent}{0pt}
    \setlength{\parskip}{6pt plus 2pt minus 1pt}}
}{% if KOMA class
  \KOMAoptions{parskip=half}}
\makeatother
\usepackage{xcolor}
\setlength{\emergencystretch}{3em} % prevent overfull lines
\providecommand{\tightlist}{%
  \setlength{\itemsep}{0pt}\setlength{\parskip}{0pt}}
\setcounter{secnumdepth}{-\maxdimen} % remove section numbering
\ifLuaTeX
\usepackage[bidi=basic]{babel}
\else
\usepackage[bidi=default]{babel}
\fi
\babelprovide[main,import]{japanese}
% get rid of language-specific shorthands (see #6817):
\let\LanguageShortHands\languageshorthands
\def\languageshorthands#1{}
\ifLuaTeX
  \usepackage{selnolig}  % disable illegal ligatures
\fi
\IfFileExists{bookmark.sty}{\usepackage{bookmark}}{\usepackage{hyperref}}
\IfFileExists{xurl.sty}{\usepackage{xurl}}{} % add URL line breaks if available
\urlstyle{same}
\hypersetup{
  pdftitle={二次数式パターン},
  pdflang={ja},
  hidelinks,
  pdfcreator={LaTeX via pandoc}}

\title{二次数式パターン}
\author{}
\date{}

\begin{document}
\maketitle

\hypertarget{ux4e8cux6b21ux30d1ux30bfux30fcux30f3}{%
\section{二次パターン}\label{ux4e8cux6b21ux30d1ux30bfux30fcux30f3}}

\begin{quote}
一部の構造による二次因数分解
\end{quote}

二次パターンとは因数分解の前に二次方程式に特定する必要がある。

その方程式を特定される必要があります。また、いくつかパターンがある。そして、こちらはそのいくつか二次パターンです。

\hypertarget{ux65b9ux7a0bux5f0fux4f8b}{%
\subsection{方程式例}\label{ux65b9ux7a0bux5f0fux4f8b}}

こちらは二次方程式の例です。パターンを探して特定して下さい。

\begin{enumerate}
\setcounter{enumi}{-1}
\tightlist
\item
  第零歩:二次方程式を特定する。

  \[(x - 1)^{2} + (x - 1) + 4\]
\item
  第一歩:二次方程式を解ける。

  \begin{enumerate}
  \item
    \[x^{2} - 2x + 1 + x - 1 + 4\]
  \item
    \[x^{2} - 2x + x + 1 - 1 + 4\]
  \item
    \[x^{2} - x + 4\]
  \end{enumerate}
\item
  第二歩:パターンを特定する。
  「{\((U + V)^{2}\)}や{\((U - V)^{2}\)}か{\((U + V)(U - V)\)}」

  \begin{itemize}
  \tightlist
  \item
    この数式が解答がありない。
  \end{itemize}
\end{enumerate}

\end{document}
